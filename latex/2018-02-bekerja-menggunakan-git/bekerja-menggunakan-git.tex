\documentclass{article}

\usepackage{graphicx}
\usepackage{float}
\usepackage[bahasa]{babel}
\usepackage{hyperref}
\usepackage{listings}
\usepackage{xcolor}

\usepackage{fancyhdr}
\setlength{\headheight}{15.2pt}
\pagestyle{fancy}

\lhead{oleh: E.R. Nurwijayadi S.T.}
\rhead{halaman \thepage}
% \cfoot{center of the footer!}
\renewcommand{\headrulewidth}{0.4pt}
\renewcommand{\footrulewidth}{0.4pt}

\title{\textbf{Menggunakan Git untuk Bekerja dengan Beberapa Komputer}}
\date{2018-01-15}
\author{E.R. Nurwijayadi S.T.}


\hypersetup{
    colorlinks=true,       % false: boxed links; true: colored links
    linkcolor=blue,        % color of internal links
}

\lstdefinestyle{BashInputStyle}{
  language=bash,
  basicstyle=\small\sffamily,
  numbers=none,
  numberstyle=\tiny,
  numbersep=3pt,
  frame=tb,
  columns=fullflexible,
  backgroundcolor=\color{gray!15},
  linewidth=1.1\linewidth,
  xleftmargin=0.0\linewidth
}

\begin{document}

\maketitle
\tableofcontents
\listoffigures
% \listoftables

\pagenumbering{gobble}
\newpage

\pagenumbering{arabic}

% -- -- -- -- -- -- -- -- -- -- -- --

\section{Kata Pengantar}

\begin{quote}
Berkas ini dapat juga dijadikan panduan bagi pemula.
\end{quote}

Penulis telah sekian lama bekerja menggunakan \textbf{git}
untuk menulis artikel dalam blog pribadi, dengan menggunakan satu notebook.
Tiba saatnya penulis beranjak dari bekerja menggunakan satu notebook,
menjadi bekerja di mana saja, kapan saja.
Dalam artikel ini, penulis menggunakan contoh kasus
bekerja antara rumah dan kantor, yaitu

\begin{itemize}
\item Komputer pertama di rumah dan,
\item Komputer kedua di kantor.
\end{itemize}

Tidak perlu lagi menjinjing notebook kemana-mana.
Dari data terpusat di satu notebook, menjadi data terpusat di github,
untuk dipakai di berbagai tempat.

Berikut langkah-langkah yang perlu dicapai.

\begin{itemize}
\item Membikin Repository: Di akun github
\item Komputer Pertama: \textbf{push} untuk \textbf{commit} yang pertama kali, dengan komputer pertama.
\item Segarkan Perambah: Refresh browser, untuk memeriksa hasilnya.
\item Komputer Kedua: \textbf{pull} repository yang baru dibikin, dengan komputer kedua, dan melakukan perubahan.
\item Kembali ke komputer Pertama: \textbf{pull} perubahan yang baru saja dibikin, dengan komputer pertama.
\end{itemize}

\begin{quote}
Jangan lupa berdoa, sebelum memulai segala sesuatunya.
\end{quote}

\subsection{Referensi}

Berkas ini adalah bentuk terjemahan dari artikel berbahasa Inggris yang sebelumnya telah ditulis di dalam Blog.

\href{http://epsi-rns.github.io/opensource/2018/01/15/using-git.html}{Using Git for Working Between Computer}

% -- -- -- -- -- -- -- -- -- -- -- --
\newpage

\section{Membuat Repository}

\begin{quote}
\textbf{push} untuk \textbf{commit} yang pertama kali, dengan komputer pertama.
\end{quote}

Pembaca harus melakukan secara manual.

\begin{itemize}
\item Membikin akun di \textbf{\href{https://github.com}{github.com}}, Lalu
\item Membikin repository, misalnya kita beri nama \textbf{anycode}
\end{itemize}

Hasilnya akan tampak sebagaimana berikut di bawah:

\begin{figure}[H]
  \includegraphics[width=\linewidth]{github-create-account.png}
  \caption{Membuat Repository.}
  \label{fig:git.create}
\end{figure}

% Gambar \ref{fig:git.create} Membuat Repository.

\begin{quote}
Satu langkah kecil, untuk loncatan yang lebih besar.
\end{quote}

% -- -- -- -- -- -- -- -- -- -- -- --
\newpage

\section{Komputer Pertama: Rumah}

\begin{quote}
\textbf{push} untuk \textbf{commit} yang pertama kali, dengan komputer pertama.
\end{quote}

\newcommand{\shellcmd}[1]{\indent\indent\texttt{\$ #1}\\}

\shellcmd{git push ...}

Ikuti saja langkah di bawah, yaitu panduan yang umum dari github.

\begin{itemize}
\item Bikinlah directory dengan nama apa saja sebagai directory kerja kita,
misalnya kita beri nama \textbf{anycode}
\item Lalu bikinlah berkas \textbf{README.md} dengan isi bebas.
Isilah dengan apa saja yang terlintas di kepala pembaca.
Silahkan memakai \textit{text editor} apa saja yang disukai misalnya:
\textbf{geany}, \textbf{gedit}, \textbf{kate}, \textbf{emacs} atau \textbf{ViM}.
\end{itemize}

\newcommand*{\Package}[1]{\texttt{#1}}%

\begin{lstlisting}[style=BashInputStyle]
    % cd githublab

    % mkdir anycode

    % cd anycode

    % touch README.md

    % geany README.md

    % git init
    Initialized empty Git repository in /home/epsi/githublab/anycode/.git/
\end{lstlisting}

Bilamana mana langah ini sudah dilalui dengan benar,
dan \textbf{README.md} telah berisi sesuatu, maka lanjutkanlah dengan bersemangat.

\begin{lstlisting}[style=BashInputStyle]
    % git add README.md

    % git status
    On branch master

    Initial commit

    Changes to be committed:
      (use "git rm --cached <file>..." to unstage)

    	new file:   README.md

    % git commit -m 'I love you, honey!'
    [master (root-commit) 6119e49] I love you, honey!
     1 file changed, 2 insertions(+)
     create mode 100644 README.md
\end{lstlisting}

Dalam berbagai tutorial di internet kita biasanya memakai
kata \textbf{origin} sebagai nama dari \textit{remote repository}.
Namun sebetulnya \textit{remote repository} ini dapat dinamai sesuka hati kita,
misalnya \textbf{cinta}.

\begin{lstlisting}[style=BashInputStyle]
    % git remote add cinta https://github.com/epsi-rns/anycode.git

    % git push --set-upstream cinta master
    Username for 'https://github.com': epsi-rns
    Password for 'https://epsi-rns@github.com': 
    Counting objects: 3, done.
    Delta compression using up to 2 threads.
    Compressing objects: 100% (2/2), done.
    Writing objects: 100% (3/3), 270 bytes | 0 bytes/s, done.
    Total 3 (delta 0), reused 0 (delta 0)
    To https://github.com/epsi-rns/anycode.git
     * [new branch]      master -> master
    Branch master set up to track remote branch master from cinta.
\end{lstlisting}

\begin{figure}[H]
  \includegraphics[width=\linewidth]{gitcli-first-process-account.png}
  \caption{Melakukan Push di komputer Pertama.}
  \label{fig:git.first}
\end{figure}

% Gambar \ref{fig:git.first} Komputer Pertama.

\newpage

Perhatikan:

\shellcmd{git push -u cinta master}

Arti perintah di atas adalah kita mendorong isi repository local
ke \textit{remote} yang kita tandai bernama \textbf{cinta},
tepatnya ke \textbf{branch} di github yang bernama \textbf{master}.

\textbf{branch} di github yang bernama \textbf{master} ini,
telah ada secara default saat repository dibikin.

% -- -- -- -- -- -- -- -- -- -- -- --

\section{Segarkan Perambah (Refresh Browser)}

\begin{quote}
Segarkan Perambah, untuk memeriksa hasilnya.
\end{quote}

Mari kita lihat hasilnya di \textbf{github}.
Kira-kira akan tampak sebagai berikut:

\begin{figure}[H]
  \includegraphics[width=\linewidth]{github-refresh-browser.png}
  \caption{Menyegarkan Perambah.}
  \label{fig:git.refresh}
\end{figure}

% Gambar \ref{fig:git.refresh} Segarkan Perambah.

% -- -- -- -- -- -- -- -- -- -- -- --
\newpage

\section{Komputer Kedua: Kantor}

\begin{quote}
{pull} repository yang baru dibikin, dengan komputer kedua, dan lakukan perubahan.
\end{quote}

Sekarang berpindahlah ke komputer lain.
Untuk melakukan uji coba, sebetulnya tidak harus menggunakan komputer lain,
namun cukup memakai directory lain.

\begin{itemize}
\item Bikinlah directory dengan nama apa saja sebagai directory kerja yang baru,
misalnya kita beri nama \textbf{mycode}
\item Kemudian lakukan \textbf{pull}.
\item Lalu ubahlah isi repository, lakukan \textbf{commit}, lalu \textbf{push}
\end{itemize}

Pada saat melakukan \textbf{pull}, sekali lagi tidak harus
memakai kata \textbf{origin}, namun dapat dinamai sesuka hati kita,
misalnya \textbf{rindu}.

\begin{lstlisting}[style=BashInputStyle]
    % mkdir mycode

    % cd mycode

    % git init
    Initialized empty Git repository in /home/epsi/githublab/mycode/.git/

    % git remote add rindu https://github.com/epsi-rns/anycode

    % git remote --verbose
    rindu	https://github.com/epsi-rns/anycode (fetch)
    rindu	https://github.com/epsi-rns/anycode (push)

    % git pull rindu master
    remote: Counting objects: 3, done.
    remote: Compressing objects: 100% (2/2), done.
    remote: Total 3 (delta 0), reused 3 (delta 0), pack-reused 0
    Unpacking objects: 100% (3/3), done.
    From https://github.com/epsi-rns/anycode
     * branch            master     -> FETCH_HEAD
     * [new branch]      master     -> rindu/master
     % ls -l
    total 4
    -rw-r--r-- 1 epsi users 56 Jan 23 03:10 README.md
\end{lstlisting}

\newpage

Ubahlah sesuatu di dalam repository, misalnya dengan
mengubah isi \textbf{README.md}. 

\begin{lstlisting}[style=BashInputStyle]
    % geany README.md

    % touch TO-DO.md

    % geany TO-DO.md
\end{lstlisting}

\begin{figure}[H]
  \includegraphics[width=\linewidth]{gitcli-second-change-content.png}
  \caption{Mengubah Isi di Komputer Kedua.}
  \label{fig:git.second}
\end{figure}

% Gambar \ref{fig:git.second} Komputer Kedua.

\newpage

Kemudian lakukan \textbf{commit}
dengan keterangan yang jelas misalnya \textbf{beli bunga}.
Kemudian lakukan \textbf{pull} ke \textbf{github}.

\begin{lstlisting}[style=BashInputStyle]
    % git add --all

    % git commit -m 'TO DO: Beli Bunga'
    [master 5c517cc] TO DO: Beli Bunga
     2 files changed, 2 insertions(+), 1 deletion(-)
     create mode 100644 TO-DO.md
     % git push -u rindu master
    Username for 'https://github.com': epsi-rns
    Password for 'https://epsi-rns@github.com': 
    Counting objects: 4, done.
    Delta compression using up to 2 threads.
    Compressing objects: 100% (3/3), done.
    Writing objects: 100% (4/4), 370 bytes | 0 bytes/s, done.
    Total 4 (delta 0), reused 0 (delta 0)
    To https://github.com/epsi-rns/anycode
       6119e49..5c517cc  master -> master
    Branch master set up to track remote branch master from rindu.
\end{lstlisting}

\begin{quote}
Periksa hasil perubahannya di perambah.
\end{quote}

% -- -- -- -- -- -- -- -- -- -- -- --
\newpage

\section{Kembali ke Komputer Pertama: Rumah}

\begin{quote}
\textbf{pull} perubahan yang baru saja dibikin, dengan komputer pertama.
\end{quote}

Sekarang, kembalilah ke komputer pertama ke dalam \textbf{anycode} directory.

Untuk melihat perubahan yang terkini dar github, cukup hanya dengan melakukan
\textbf{pull}.

\begin{lstlisting}[style=BashInputStyle]
    % cd anycode

    % git pull
    remote: Counting objects: 4, done.
    remote: Compressing objects: 100% (3/3), done.
    remote: Total 4 (delta 0), reused 4 (delta 0), pack-reused 0
    Unpacking objects: 100% (4/4), done.
    From https://github.com/epsi-rns/anycode
       6119e49..5c517cc  master     -> cinta/master
    Updating 6119e49..5c517cc
    Fast-forward
     README.md | 2 +-
     TO-DO.md  | 1 +
     2 files changed, 2 insertions(+), 1 deletion(-)
     create mode 100644 TO-DO.md
\end{lstlisting}

Repository di komputer pertama (local) sudah diperbarui.


\begin{figure}[H]
  \includegraphics[width=\linewidth]{gitcli-first-pull.png}
  \caption{Melakukan Pull di komputer Pertama.}
  \label{fig:git.first.pull}
\end{figure}

% Gambar \ref{fig:git.first.pull} Kembali ke Komputer Pertama.

\begin{lstlisting}[style=BashInputStyle]
    % ls -l
    total 8
    -rw-r--r-- 1 epsi users 70 Jan 23 04:06 README.md
    -rw-r--r-- 1 epsi users 18 Jan 23 04:06 TO-DO.md
\end{lstlisting}

% -- -- -- -- -- -- -- -- -- -- -- --

\section{Penutup}

Terimakasih telah membaca.
Semoga bermanfaat.




\end{document}
